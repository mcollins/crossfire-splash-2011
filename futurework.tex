\section {Future Work}
\subsection {Web Sockets}
Implementations of the WebSocket API\cite{websocketapi} and
protocol\cite{websocketprotocol} standard are beginning to appear in recent
versions of several Web Browsers. The WebSocket API allows JavaScript code in a
Web page to create a full-duplex socket connection to another host using a
lightweight protocol. Using these API's, it may be possible in the future to
provide a Firebug front-end (similar to Firebug Lite) and Crossfire client which
do not rely on other browser-specific extension APIs.
\subsection {Mobile Web Debugging}
Mobile devices such as smart phones and tablet computers now include
full-featured Web browsers. In contrast to Desktop browsers, which have
been adding more tools for developers, mobile browsers do not have the built-in
Web development tools. The intuitive reason for this lack of tools is that the
form factors of these mobile devices do not lend themselves to software
development tasks. In this scenario, the remote debugging solution may be the
only viable alternative.

\subsection {Multi-user Debugging}
The architecture and system we have built thus far has been implemented and
demonstrated with a single user in mind, but is not restricted to that. Even in
cases with a single user, there could be cases where it is desirable for a
Crossfire server to support connections to multiple clients, such has connecting
an external IDE while also using Firebug's in-browser UI.

Though the default operating mode of our Crossfire server implementation is to
accept incoming connections only on the local host interface, it is possible to
connect from a remote host. Since Crossfire should support multiple clients, it
is conceivable that multiple users could use separate Crossfire clients
connected to the same Web page instance to collaborate on developing or
debugging that page. While it is beyond the original goals of the Crossfire
project, it would be possible to build on the Crossfire work to add additional
features to facilitate this kind of collaborative debugging.
