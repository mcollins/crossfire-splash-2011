\section{Design Motivation}
As a practical project supporting 3 million users,  Firebug drove many of the design considerations behind \textit{Crossfire}. These considerations are a mixture of purely technical issues and issues of open source project management. The main design goals, multiprocess support, remote and mobile debug, and open Web, cross-browser debugging lead to many of the technical design choices. On the project management side we must work with development resources motivated by goals: no matter how much value Firebug users may recieve from a goal, the selection must be limited by the motivation of open source contributors.

Necessity motivated first \textit{Crossfire} design goal, multiprocess support. Soon after the Google Chrome browser was released, the Firefox team at Mozilla began plans to convert Firefox to a multi-processor design.  The Google browser used one controlling process for the application and one process for each Web page.  This allows the browser to use the operating system isolation to prevent problems on one page from bringing down the entire application and it allows each page to use a different physical processor on modern multi-core computers \cite{GoogleChrome}.  Depending upon the Firefox browser plaform changes, a shift to multiprocess could render Firebug unusable. As a practical matter we could not wait to for the new platform to be available: with more than 50kloc of code, only a few full-time developers, and a commitment to continous compatibility with Firefox we had to begin work immediately to insure that our small resource could complete the transition in time to remain a viable project. Therefore we assumed that Firefox would adopt an architecture similar to Google Chrome: a client/server split debugger with a backend in one Web page process and a front end in another process.  We believe that this assumption is planning for the worst case: converting Firebug to client/server is a multi-person-year effort but very likely to work with what ever the Firefox team decides to do.

While necessity forced our action, opportunity followed. The client/server choice, if successful, adds two new dimensions to Firebug for users: remote debug and mobile device debug. We expect the value of these dimensions to grow as more developers work in distributed teams and as mobile plays an increasingly important role in Web application development.  In fact this value was recognized by the DragonFly Web debugger for Opera well before even the Google Chrome browser.   The additional cost of designing for remote and mobile debug on top of a client/server design, primarily mechanisms for specifying the connection addresses, comes with high potential benefit.  Moreover, the benefit aligns with directions important to the projects primary open source contributors.

The final goal, open Web, cross-browser debugging, offers even more benefits to Firebug users.  Web application developers by definition target all Web users, but the all Web users are not running identical Web platforms. Almost all potential users of a Web site will be running one of few similar but slightly different browsers. The commonality allows Web developers to do most of their work on one browser, then test for differences on other browsers. Of course when this fails, they need to debug the problem on a browser with unfamilar debugging tools. A common debugging tool across the major browsers would help with this common and significant problem.

The benefit of cross-browser debugging comes at a steep cost for the project. Instead of one server and one client, we face at minimum one server for every browser. And for each server we have to deal with both the slight differences in browser implementation of standard Web APIs and potential large differences in how debuggers can connect to the browser. In addition this goal implies that the client and the communication protocol should be built from open web standards to maximize the reuse across servers. 

Perhaps unique to open source projects, Firebug might balance the cost of implementing cross-browser debugging support by attracting more contributors interested in this particular goal. That is, by adding this costly goal we can attract new contributors, allowing us to create more total value. In particular new contributors from the Orion project\cite{orion}, joined to create \textit{Crossfire} server for Microsoft Internet Explorer and from the Eclipse project\cite{EclipseJSDT} to create new \textit{Crossfire} client in Java for connecting to Eclipse.

These design goals created constraints for \textit{Crossfire}. Above we outlined how the multiprocess support lead to a client-server design choice. Support for remote and mobile debug forces isolation of user interface to the client (excepting some small interface for connection specification). The cross-browser goal creates constraints indirectly: to minimize the extra cost of supporting multiple servers we chose to adopt the Google Chrome communications channel (sockets) and wire protocol format (JSON). Neither Firefox nor Internet Explorer had existing servers, so they did not alter our choices. Opera had a server but no one on the open source team planned to work with Opera and the server itself was not open source making implementation more difficult.  Since Firebug is already written in JavaScript, JSON format is especially easy to work with and has good performance\cite{JSON}.  For the communications protocol, HTTP would be a better choice for the project: the JavaScript support for HTTP is much better than sockets and HTTP works better in practical remote scenarios through firewalls.  However we made the judgement that better socket support was coming in future\cite{WebSockets}, support was adequate now, and lowering cost on the Google Chrome server was important.