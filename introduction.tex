\begin{introduction}
\section{Introduction}
Web Applications continue to grow in size and complexity. The
emergence of common toolkits and libraries for Javascript along with performance
increases in Web Browsers fuels growth in client-side Web application
development. As more functionality shifts from server- to client-side, the size
of Javascript codebases naturally increases. Unfortunately for developers, the
tools to develop, manage, and debug larger codebases have not followed the
applications into the Web application development space.

Sophisticated development tools become crucial for understanding
and working with larger codebases. Compilers, debuggers, and other tools are
often combined to create an Integrated Development Environment (IDE) where
most, if not all, development tasks can be performed. Web development tools, on
the other hand, often reside in the Web Browser. In 2011, all of the major Web
browsers ship with some kind of Web development tools included\cite{something},
and still more are available as plugins or add-ons. These tools operate
directly on the runtime Web application that is loaded into the browser, and do
not retain any knowledge of or connection to the original source code. Therefore
a developer must manually apply any changes made in one of these tools to his or
her source code.

The rise of Mobile and Tablet computers which ship with fully functional Web
browsers means that Web application developers have additional form-factors to
consider.

Tools such as ``Inspectors'' and Firebug's HTML Breakpoints\cite{jjb-www2010}
enable developers to quickly locate the section of code they are interested in.
\end{introduction}