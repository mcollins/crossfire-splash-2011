
\section{Introduction}
Web Applications continue to grow in size and complexity. Asynchronous
background data download (AJAX) started the surge. This led to the emergence of
common toolkits and libraries for JavaScript, which drove  performance increases
in Web Browsers, fueling more growth in client-side Web application development.
These improvements, combined with new features available in Web browsers shifted
investment from server- to client-side. Recent empirical analysis of
representative
 major Web sites shows program sizes in the range of hundreds of kilobytes of
sophisticated code.\cite{VitekDynamicJS2010}

To develop and maintain these large applications, programmers and designers rely
on numerous tools, most notably Web page debuggers. This paper describes \textit{Crossfire},
a project and protocol to provide a next generation platform for Web page debuggers and the major
re-architecting of the most widely used Web page debugger, Firebug, to use \textit{Crossfire}
to support its client-server communications. Our description focuses on
practical, state-of-the-art issues in an on-going, fast-moving, open-source project. Thus
we cover details of protocol important for implementation and issues of matching
resources to goals important for project management: we must deal with both
low and high level issues to succeed.

\section{Background}
To understand the importance and challenges of the \textit{Crossfire} work we
start by introducing Firebug. Released by Joe Hewitt in 2006, Firebug was the
first integrated Web debugger. Firebug is a runtime debugger: it directly
accesses, responds to, and operates on the running Web browser.  Rather than
separate views of JavaScript, CSS, and HTML, Firebug integrated its views such
that interaction with, for example, an HTML element would cause synchronized
views of the CSS rules. Rather than static
 views of browser state, Firebug included dynamics like network traffic analysis
 and console logging; rather
than read-only views, Firebug allowed live edits where possible so developers
could try out changes. The resulting tool became very popular with developers
and contributing significantly to the growth in Web applications.

The primary implementation of Firebug is a Firefox extensions, a supplemental
software component that loads into the Firefox Web browser. A secondary
implementation with fewer features and, in particular, limited support for
JavaScript debugging, called Firebug Lite, works in multiple browsers. The
success of Firebug lead to implementations of Web Page debuggers in other
browsers, including DragonFly for Opera\cite{DragonFly}, Web Inspector for Google Chrome and
Apple Safari\cite{WebInspector}, and the developer tools in Microsoft Internet Explorer\cite{IETools}. 
Since 2007
Firebug has been developed as an open source project, with seven major releases.

To give a flavor of the kinds of operations Firebug supports, we outline  an
example more completely described in Ref.\cite{jjb-www2010}. Suppose a developer
wants to understand why a block of text in the Web page turned green while the
page was loading. They might use the Firebug "inspect" feature, causing Firebug
to display the HTML element under the mouse. They see that the element has a
\textit{style} attribute setting the color green. While hovering over the green
block of text, the developer clicks down to lock the user interface on the
element, then moves to the HTML panel in Firebug and right-clicks on the
selected HTML element representation. A menu pops up allowing the developer to
select "Break On Attribute Change". When the developer reloads the page, Firebug
halts in the JavaScript code panel, on the line where the attribute is changed
from red to green.

This example illustrates that we will need to synchronize mouse events on the
Web page with the debugger UI, identify HTML element representations rendered in
the debugger UI with the elements in the Web page and synchronize DOM mutation
event handling wit JavaScript execution. 
Crossfire is designed to support the kinds of features available in current built-in
development tools such as Firebug, while enabling the advantages of a remote
debug connection.


