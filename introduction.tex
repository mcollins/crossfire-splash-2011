
\section{Introduction}
Web Applications continue to grow in size and complexity. Asynchronous background 
data download (AJAX) started the surge. 
This led to the 
emergence of common toolkits and libraries for Javascript which drove  performance
increases in Web Browsers fueling more growth in client-side Web application
development. These improvements, combined with new features available in Web browsers 
shifted investment from server- to client-side. Recent empirical analysis of representative
 major Web sites shows program sizes in the range of hundreds of kilobytes of
sophisticated code.\cite{VitekDynamicJS2010}

To develop and maintain these large applications, programmers and designers rely on 
numerous tools, most notably Web page debuggers. 
This paper describes a major re-architecting of the most widely used Web page debugger, 
Firebug, as a client/server system and in particular the 
\textit{Crossfire} protocol designed to support its client-server communications.  
Our description focuses on practical, state-of-the-art issues in an on-going and fast-moving project. 
Thus we cover details of protocol important for implementation and issues of 
matching resources to goals important for project management: we must deal with both 
extremes to succeed.

\section{Background}
To understand the importance and challenges of the \textit{Crossfire} work we start by introducing Firebug.
Released by Joe Hewitt in 2006, Firebug was the first integrated Web debugger. Firebug is a runtime 
debugger: it directly accesses, responds to, and operates on the running Web browser.  Rather than separate 
views of JavaScript, CSS, and HTML, Firebug integrated its views such that interaction with, 
for example, an HTML element would cause synchronized views of the CSS rules. Rather than static 
 views of browser state, Firebug included dynamics like network traffic analysis and console logging; rather 
than read-only views, Firebug allowed live edits where possible so developers could try out changes.
The resulting tool became very popular with developers and contributing significantly to the growth in Web applications.

The primary implementation of Firebug is a Firefox extensions, a supplemental software 
component that loads into the Firefox Web browser. A secondard implementation with fewer features and, 
in particular, limited support for JavaScript debugging, called Firebug Lite works in multiple browsers. 
The success of Firebug lead to implemenations of Web Page debuggers in other browsers, including 
DragonFly for Opera, Web Inspector for Google Chrome and Apple Safari, and the developer tool in 
Microsoft Internet Explorer. Since 2007 Firebug has been developed as an open source project, with 
seven major releases.

To give a flavor of the kinds of operations Firebug supports, we outline two examples more completely
described in Ref.\cite{jjb-www2010}. First suppose a developer wants to understand why a block of text 
in the Web page turned green while the page was loading. More. 

Second, 

(Refactoring timeline)