
\section{Related Work}
\subsection{Web Development Tools}
Tools for developing software to run in a Web browser fit broadly into two
categories: external (to the
browser) tools, and those that are built-in to the browser.
Built-in tools include Firebug and Mozilla Firefox's developer tools,
Opera's DragonFly, Internet Explorer's Developer Tools, Google Chrome's Dev
Tools and WebKit's Web Inspector.

\subsubsection{Weinre}
The Weinre (Web Inspector Remote)\cite{wenire} project implements a debugger for
web pages that is designed to work remotely, reusing the user interface code
from WebKit's Web Inspector.

\subsubsection{Eclipse JSDI}
\subsubsection{Orion}
\subsection{Remote Protocols}
Many protocols have already been designed for the purposes of remotely debugging
an application running in another process, virtual machine, host, etc. The GNU
GDB debugger has an associated Remote Serial Protocol (RSP)\cite{gdb-rsp}. While
it is the only debugger we know of with its own song \cite{gdb-song}, it is
primarily designed for debugging code running on embedded systems, and would not be
well-suited for use with Firebug.
\subsubsection{JNDI/JDWP}
Java does remote debugging too.
\subsubsection{DBGp}
DBGp\cite{dbgp}, is an acronym for Debug Protocol, and was developed for version
2 of the XDebug debugger for the PHP language. Although it was designed not to
be language specific, many of the commands are intended to be synchronous, as
opposed to the asynchronous nature of Crossfire. DBGp also allows for the
debugger engine to send an event to a client via the 'notification' element,
with a custom body. However in order to support Firebug, we would need to define
the same events defined by Crossfire as DBGp notifications.
\subsubsection{Opera Scope Protocol}
The Opera browser has a built-in Web development tool called DragonFly,
which also supports the Scope remote debug protocol. The Scope protocol supports
XML and JSON formats, and features such as JavaScript debugging and remote DOM
inspection.
\subsection{Web Inspector Protocol}
\cite{opera-scope}

