
\section{Related Work}
\subsection{Web Development Tools}
Tools for developing software to run in a Web browser fit broadly into two
categories: external (to the
browser) tools, and those that are built-in to the browser.
Built-in tools include Firebug and Mozilla Firefox's developer tools,
Opera's DragonFly, Internet Explorer's Developer Tools, Google Chrome's Dev
Tools and WebKit's Web Inspector.

\subsubsection{Weinre}
The Weinre (Web Inspector Remote)\cite{weinre} project implements a debugger for
web pages that is designed to work remotely, reusing the user interface code
from WebKit's Web Inspector. The project was an experiment to get the Web
Inspector interface to run in a Web page, much like Firebug Lite. Since Weinre
supports remote inspection, and not JavaScript debugging, it could also be complimentary
to our Crossfire work.

\subsubsection{Eclipse JSDT}

\subsubsection{Orion}

\subsection{Remote Protocols}
Many protocols have already been designed for the purposes of remotely debugging
an application running in another process, virtual machine, host, etc. The GNU
GDB debugger has an associated Remote Serial Protocol (RSP)\cite{gdb-rsp}. While
it is the only debugger we know of with its own song \cite{gdb-song}, it is
primarily designed for debugging code running on embedded systems, and would not
be well-suited for use with Firebug. The Java Debug Wire Protocol (JDWP)
\cite{jdwp} provides remote debugging of Java Virtual Machines. The protocol
supports command and response message pairs similar to Crossfire and other Web
debugging protocols, however the design of the protocol
\subsubsection{DBGp}
DBGp\cite{dbgp}, is an acronym for Debug Protocol, and was developed for version
2 of the XDebug debugger for the PHP language. Although it was designed not to
be language specific, many of the commands are intended to be synchronous, as
opposed to the asynchronous nature of Crossfire. DBGp also allows for the
debugger engine to send an event to a client via the 'notification' element,
with a custom body. However in order to support Firebug, we would need to define
the same events defined by Crossfire as DBGp notifications, essentially creating
another protocol within the protocol.

\subsubsection{Opera Scope Protocol}
The Opera browser has a built-in Web development tool called DragonFly, which
also supports the Scope remote debug protocol. The Scope
protocol\cite{opera-scope} supports XML and JSON formats, and features such as
JavaScript debugging and remote DOM inspection. It is used to allow the desktop
DragonFly client to connect to another Opera browser instance, including mobile
versions.

\subsubsection{V8 / Chrome Dev Tools Protocol}
The V8 protocol\cite{v8} is a JSON-based wire protocol for debugging JavaScript
programs running within the V8 engine. The Chrome Dev Tools
protocol\cite{chrome-dev-tools} wraps the V8 protocol to provide the additional
information needed to debug a Web page running within Google's Chrome browser.
The protocol implements JSON messages over TCP/IP sockets, and was the basis for
much of the initial work on the Crossfire protocol. However, over the course of
developing Crossfire and refactoring of Firebug, it was realized that Crossfire
would require more functionality than the Chrome protocols provided.


