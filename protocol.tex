\section {Protocol}
\subsection {Overview}
The Crossfire protocol is an asynchronous, bi-directional protocol designed to
enable the full functionality of the Firebug debugger in a multi-process or
remote scenario. Where it was possible, the design of the protocol took cues
from existing debug protocols as well as common Web technologies (e.g.
HTTP\cite{http}, JSON\cite{json}). However certain features unique to Firebug
and to debugging code running inside a Web Browser have to be taken into account.

Code that the user wishes to debug may not always be running.

The user may have several user interfaces in which to interact and debug the
runtime.

\subsection {Connection and Handshake}
Crossfire does not specify a standard or well-known port. Port agreement is left
up to the user, or the client software must start the server listening on
the same port it will attempt to connect to.

The Crossfire server listens on for a TCP connection on the specified port
(greater than 1024).  A client wishing to connect sends the string
``CrossfireHandshake'' followed by a CRLF. The server replies with the same
string, at which point the connection is established and the client may begin
sending requests and receiving events from the server.

\subsection {Message Packets}
A well-formed Crossfire packet contains one or more headers consisting of the
header name, followed by a colon (``:``), the header value, and terminated by a
CRLF. A ``Content-Length'' header containing the number of characters in the
message body is required.

The message body is separated from the headers by a blank line (CRLF), followed
by a well-formed JSON string, and terminated by a CRLF.  The message must
contain a ``type'' field with the value one of ``request'', ``response'', or
``event'', and a ``seq'' field which contains the sequence number of the packet.

Example:
TODO example


\subsection {Extensibility}

\subsection {Contexts}
A context in Firebug represents a single Web page.

\subsection {Breakpoints}
Breakpoint debugging is a standard tool for debugging software at runtime in
many languages and environments. The Web Browser environment creates several
challenges for designing a remote protocol which supports breakpoint debugging.
Firebug also introduces several types of breakpoints which are not present in
other environments \cite{jjb-www2010}
